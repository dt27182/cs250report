\section{Introduction}
In the existing automatic pipelining tool, the designer can specify the exact placement of the pipeline registers by labeling particular Chisel nodes as pipeline boundary nodes. Alternatively, the designer can label a small subset of the chisel node graph with pipeline stage numbers and have the automatic pipelining tool infer where to place the pipeline registers. In the second method of pipeline specification, the tool picks some legal pipeline register placement. A pipeline register placement is legal if every combinational logic node has all of its inputs in the same pipeline stage. However, this pipeline specification does not consider the consequences of the pipeline placement on the critical path delays.

In this paper, we will present the systems to infer not only a legal pipeline register placement, but also a pipeline register placement that optimizes the critical path delay, producing pipelined designs. In the systems, the Chisel delay backannotator annotates a Chisel graph with delay information from logic synthesis, and using the backannotated graph, the automatic pipelining tool finds the optimal placement of the pipeline registers through simulated annealing. This allows the designer to be able to create a well-balanced pipelined design without repeatedly going through the VLSI design tools to get feedback on critical path length. Combined with the ability of the automatic pipelining tool to automatically generate control logic, the designer will be able to very quickly explore the design space of different pipeline depths and hazard. 

